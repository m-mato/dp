%%%%%%%%%%%%%%%%%%%%%%%%%%%%%%%%%%%%%%%%%%%%%%%%%%%%%%%%%%%%%%%%%%%%
%% I, the copyright holder of this work, release this work into the
%% public domain. This applies worldwide. In some countries this may
%% not be legally possible; if so: I grant anyone the right to use
%% this work for any purpose, without any conditions, unless such
%% conditions are required by law.
%%%%%%%%%%%%%%%%%%%%%%%%%%%%%%%%%%%%%%%%%%%%%%%%%%%%%%%%%%%%%%%%%%%%

\documentclass[
  printed, %% This option enables the default options for the
           %% printed version of a document. Replace with `digital`
           %% to enable the default options for the digital version
           %% of a document.
  table,   %% Causes the coloring of tables. Replace with `notable`
           %% to restore plain tables.
  lof,     %% Prints the List of Figures. Replace with `nolof` to
           %% hide the List of Figures.
  lot,     %% Prints the List of Tables. Replace with `nolot` to
           %% hide the List of Tables.
  %% More options are listed in the user guide at
  %% <http://mirrors.ctan.org/macros/latex/contrib/fithesis/guide/mu/fi.pdf>.
]{fithesis3}

%% The following section sets up the locales used in the thesis.
\usepackage[
  main=slovak, %% By using `czech` or `slovak` as the main locale
               %% instead of `english`, you can typeset the thesis
               %% in either Czech or Slovak, respectively.
  slovak, english %% The additional keys allow
]{babel}          %% foreign texts to be typeset as follows:
				  %%  \begin{otherlanguage}{english}  ... \end{otherlanguage}

\usepackage{paratype}

%% The following section sets up the metadata of the thesis.
\thesissetup{
    date          = \the\year/\the\month/\the\day,
    university    = mu,
    faculty       = fi,
    type          = mgr,
    author        = Matej Majdiš,
    gender        = m,
    advisor       = doc. RNDr. Vlastislav Dohnal\, Ph.D.
    title         = {Rozpoznanie užívateľa na základe informácií o HTTP komunikácií},
    TeXtitle      = {Rozpoznanie užívateľa na základe informácií o HTTP komunikácií},
    keywords      = {keyword1, keyword2, ...},
    TeXkeywords   = {keyword1, keyword2, \ldots},
}

\thesislong{abstract}{
	TODO

}

\thesislong{thanks}{
	TODO
}

%% The following section sets up the bibliography.
\usepackage{csquotes}
\usepackage[              %% When typesetting the bibliography, the
  backend=biber,          %% `numeric` style will be used for the
  style=numeric,          %% entries and the `numeric-comp` style
  citestyle=numeric-comp, %% for the references to the entries. The
  sorting=none,           %% entries will be sorted in cite order.
  sortlocale=auto         %% For more unformation about the available
]{biblatex}               %% `style`s and `citestyles`, see:
%% <http://mirrors.ctan.org/macros/latex/contrib/biblatex/doc/biblatex.pdf>.
\addbibresource{example.bib} %% The bibliograpic database within
                          %% the file `example.bib` will be used.
                          
\usepackage{makeidx}      %% The `makeidx` package contains
\makeindex                %% helper commands for index typesetting.

%% These additional packages are used within the document:
\usepackage{paralist}
\usepackage{amsmath}
\usepackage{amsthm}
\usepackage{amsfonts}
\usepackage{url}
\usepackage{menukeys}

\begin{document}

\chapter{Úvod}
Problematika jednoznačnej identifikácie používateľa je dnes veľmi
dôležitou a riešenou témou. Jedným z hlavných dôvodov je fakt, že väčšina
dnešných existujúcich, prípadne novo vznikajúcich systémov a aplikácií je
nejakým spôsobom zapojená do Internetu. Zároveň zaznamenávame nárast
aplikácií, ktoré poskytujú užívateľom webové rozhranie a ústup takzvaných
desktopových aplikácií.

---------------------------------
OBR. 1 - WEB apps vs DESKTOP apps
---------------------------------

	Z tohto vyplýva potreba rozoznania a identifikácie "používateľov", ktorý s
danou aplikáciou interagujú. Existuje niekoľko rôznych prístupov k 
identifikácií, od mapovania IP adries sieťovej vrstvy až po aplikačnú správu
užívateľských účtov. Podrobne sa nimi zaoberá [kapitola ~Existujúce prístupy].
Cieľom tejto práce je vytvoriť unikátny identifikátor na základe informácií
dostupných z HTTP protokolu. Pred zostavením samotného algoritmu je preto
dôležité popísať niektoré kľúčové oblasti a postupy.
Nasledujúce kapitoly sa preto budú stručne zaoberať fungovaním aplikácií typu
klient-server, modelom sieťových vrstiev či útokmi typu Denial of Service.
Ďalej v práci popíšem spomínané existujúce prístupy a vlastný návrh algoritmu 
identifikácie užívateľa. 

\section{Aplikácie typu Klient-Server}
	S pokračujúcim vývojom nových technológií sa Web stáva stále väčšou súčasťou
našich životov. Web taktiež už nie je limitovaný prehliadaním na počítačoch.
Musí sa prispôsobovať rôznym novým technológiám, ako sú napríklad mobilné, či
iné zriadenia. Najčastejšie používaným modelom komunikácie pre architektúru
webových aplikácií je tzv. Klient-Server model. Základnou myšlienkou tohto modelu
je zaslanie požiadavku (requestu) klientom na server, ktorý vystupuje ako
poskytovateľ služby.

-----------------------------------
OBR. 2 - Klient-Server basic schema
-----------------------------------

\subsection{Klient-Server model}
Pretože Klient-Server model je používaný rôznymi typmi aplikácií bolo nutné použiť
štandardizované protokoly, na základe ktorý ch bude možné komunikovať. Základné
používané protokoly sú: FTP (File Transfer Protocol), Simple Mail Transfer Protocol
(SMTP) a Hypertext Transfer Protocol (HTTP). Bližšie sieťové vrstvy a jednotlivé
protokoly popisuje [kapitola ~Sieťové vrstvy].

\subsection{Architektúra}
Architektúra modelu Klient-Server sa vo všeobecnosti typicky skladá z troch 
častí:
	• Aplikačný server
	• Databázový server
	• Zariadenie klienta
Zároveň Existujú dva zakladné typy architekúr: 
	• 2-stupňová (2-tier)
	• 3-stupňová (3-tier)

2-tier architektúra zahrna len zariadenie klienta a databázový server. U tohoto
typu architektúry je aplikácia spustená na zariadení klienta, ktoré sa následne
pripája priamo na server. Zariadenie tak obsluhuje zároveň business logiku aj
zobrazovanie aplikácie. Inak tento typ architektúry nazývame aj tučný klient
(thick client).

---------------------------
OBR. 3 - Thick Client image
---------------------------

3-tier architektúra, ktorou sa budem zaoberať v tejto práci sa od 2-tier líši najmä
tým, že okrem zariadenia klienta a databázového servera zahŕňa aj aplikačný server.
Tento je následne používaný na obsluhu business logiky aplikácie a komunikáciu s
databázou, pričom zariadenie klienta slúž len na zobrazovanie. Iný názov pre takýto
typ architektúry je tenký klient (thin cient).

--------------------------
OBR. 4 - Thin Client image
--------------------------

\chapter{Sieťové vrstvy}
//TODO - Úvod
\section{Aplikačná vrstva}
\section{Transportná vrstva}
\section{Sieťová vrstva}
\section{Vrstva sieťového rozhrania}

\chapter{Útoky typu Denial of Service}

\chapter{Existujúce prístupy k identifikácií}
//TODO - Úvod
\section{Využitie sieťovej vrstvy}
//TODO - Úvod
\subsection{Internet Protocol (IP)}
\subsection{Nedostatky}
\section{Aplikačné identifikátory}

\chapter{Tvorba unikátneho identifikátoru}
//TODO - Úvod
\section{Možnosti protokolu HTTP}
\section{Možnosti TCP}
\section{Popis Algoritmu}

\section{Záver}

\makeatletter\thesis@blocks@clear\makeatother
\phantomsection %% Print the index and insert it into the
\addcontentsline{toc}{chapter}{\indexname} %% table of contents.
\printindex

\appendix %% Start the appendices.
\chapter{Príloha}
Here you can insert the appendices of your thesis.

\end{document}
